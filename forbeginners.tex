\documentclass{ximera}
\title{A Template for beginners}
\begin{document}
\begin{abstract}
\end{abstract}
\maketitle
This is where we put writing. You can type mathematics using dollar signs like this
 
$2x^3=16$

there are tools in TeXnic Center to help while you are figuring out how to type all the different symbols. Try to type something relevant to what you teach.

another handy thing is the percent sign. Use this to designate a section as notes. anyhting with the percent sign in front of it will not show up in pdf handouts or in online ximera activities. try to put a secret message in your source code
%write your secret message after a percent sign
Now lets put in some interactive things

let's ask a question
\begin{question} $3-6=$ $\answer{3}$
\end{question}

let's ask a multiple choice question
\begin{question} which flavor is not a part of neopolitan?
\begin{multipleChoice}
\choice{strawberry}
\choice{vanilla}
\choice[correct]{pumpkin spice}
\choice{chocolate}
\end{multipleChoice}
\end{question}
%Now let's try a word choice:

%\begin{question} Writing worksheets in LaTeX is\wordChoice[terrible,awful]{fun}!
%\end{question}

\end{document}